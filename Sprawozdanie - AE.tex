\documentclass[a4paper,11pt]{article}

\usepackage[T1]{polski}
\usepackage[utf8]{inputenc} 
\usepackage{graphicx}
\usepackage{float}
\usepackage{verbatim}
\hoffset=-3.0cm                 % Mniejszy lewy margines
\textwidth=18cm                 % szerzej
\evensidemargin=0pt

\voffset=-3cm                   % Mniejszy gorny margines
\textheight=27cm                % szerzej wzdluz

\usepackage{listings}
% Listingi
\lstdefinestyle{customc}{
	belowcaptionskip=1\baselineskip,
	breaklines=true,
	frame=L,
	xleftmargin=0pt,
	language=HTML,
	showstringspaces=false,
	basicstyle=\footnotesize\ttfamily,
	identifierstyle=\color{black}
}

\setlength{\parindent}{0pt}             % No paragraph indentation
\setlength{\parskip}{\medskipamount}    % Space between paragraphs
\raggedbottom   


\title{POLITECHNIKA WARSZAWSKA \\ WYDZIAŁ ELEKTRYCZNY \\}
\author{Karol Mąkosa}
\date{\today}

\begin{document}
	\thispagestyle{empty}
	\maketitle
	\date{}
	\section{Treść zadania}
	Napisać program umożliwiający znalezienie maksimum funkcji dopasowania jednej 
	zmiennej określonej dla liczb całkowitych w zadanym zakresie przy pomocy 
	elementarnego algorytmu genetycznego (reprodukcja z użyciem nieproporcjonalnej 
	ruletki, krzyżowanie proste, mutacja równomierna). Program powinien umożliwiać 
	użycie różnych funkcji dopasowania, populacji o różnej liczebności oraz różnych 
	parametrów operacji genetycznych (krzyżowania i mutacji). Program powinien 
	zapewnić wizualizację wyników w postaci wykresów średniego, maksymalnego 
	i minimalnego przystosowania dla kolejnych populacji oraz wykresu funkcji 
	w zadanym przedziale. 
	
	Program przetestować dla funkcji$ f(x)= -0.25x^2 + 5x + 6 $ dla $x= -1, 0, ... 21 $

	\section{Instrukcja działania programu}
		\subsection{Okno główne programu}
			\begin{figure}[H]
				\centering
				\includegraphics[scale=0.55]{okno.png}
			\end{figure}
		\newpage
		\subsection{Opis okna programu}
				\begin{enumerate}
					\item Informacje o obecnej populacji i o najlepiej przystosowanym elemencie
					\item Pola do ustawiania liczebności populacji i prawdopodobieństw -- krzyżowania i mutacji
					\item Przyciski sterujące
					\item Wykresy: obecnej populacji, minimalnego, średniego i maksymalnego przystosowania oraz informacja o iteracji algorytmu
				\end{enumerate}	
		\subsection{Zmiana ustawień programu}
			W programie okienkowym można zmieniać liczebność populacji oraz prawdopodobieństwa krzyżowania i mutacji.
			Aby zmienić funkcję lub przedział, na którym badamy funkcję, należy zmodyfikować początkowe linie pliku \texttt{Population.py}
	
		\subsection{Przykładowy sposób użycia}
		Aby poprawnie użyć programu należy:
		\begin{enumerate}
		\item Ustawić dane
		\item Zaakcpetować ustawienia wciskając przycisk -- \texttt{Zastosuj}
		\item Zaznaczyć lub odznaczyć opcję rysowania wykresów -- \texttt{rysuj wykresy}
		\item Wybrać sposób tworzenia kolejnych generacji populacji:
			\begin{enumerate}
				\item Jednorazowo, wciskając \texttt{Jedna generacja}
				\item Automatycznie, wciskając \texttt{Automatyczna generacja}
			\end{enumerate}
			\textit{Automatyczna generacja zakończy się, gdy w przeciągu 50 generacji nie zmieni się największe znalezione przystosowanie}
		\end{enumerate}
		
	\section{Opis eksperymentów}
	Eksperymenty przeprowadzone były dla funkcji $f(x) = -0.25x^2 + 5x + 6$, dla $x = -1, 0, ..., 21$. Maksymalne przystosowanie dla niej jest osiągane przy $x = 10$ i wynosi $31$. W poszczególnych etapach eksperymentów zmieniały się liczebność populacji i wartości krzyżowania mutacji
	
		\subsection{Różna liczebność populacji}
			Eksperymenty badające wpływ liczebności na wyniki algorytmu genetycznego wykonywane były przy prawdopodobieństwie mutacji równym 0,01 i prawdopodobieństwie krzyżowania równym 1. W tabeli umieszczone zostały:'minimalne, maksymalne i średnie przystosowanie danej populacji oraz największe znalezione dopasowanie oraz jego osobnik.
			\subsubsection{Populacja: 6 osobników}
				\begin{tabular}{|c|c|c|c|c|c|}
					\hline 
					iteracja &  min &  max & avg & global max &  X\\
					\hline
					1 & 6 & 30 & 17.54 & 30 & 12\\
					\hline
					2 & 6 & 30.75 & 18.38 & 30.75 & 11\\
					\hline
					3 & 0.75 & 30.75 & 15.0 & 30.75 & 11\\
					\hline
					4 & 0.75 & 22 &  9.5 & 30.75 & 11\\
					\hline
					5 & 0.75 & 24.75 & 14.83 & 30.75 & 11\\
					\hline
					6 & 0.75 & 24.75 & 8.08 & 30.75 & 11\\
					\hline
					7 & 0.75 & 24.75 & 13.42 & 30.75 & 11\\
					\hline
					8 & 0.75 & 27 & 13.79 & 30.75 & 11\\
					\hline
					9 & 0.75 & 24.75 & 8.79 & 30.75 & 11\\
					\hline
					10 & 0.75 & 24.75 & 12.79 & 30.75 & 11\\
					\hline
				\end{tabular} \\
				Rzeczywiste maksimum funkcji znaleziono po ok. 250 generacjach.
			\subsubsection{Populacja: 50 osobników}
				\begin{tabular}{|c|c|c|c|c|c|}
					\hline 
					iteracja &  min &  max & avg & global max &  X\\
					\hline
					1 & 0.75 & 31 & 19.43 & 31 & 10\\
					\hline
					2 & 0.75 & 31 & 18.7 & 31 & 10\\
					\hline
					3 & 0.75 & 31 & 19.25 & 31 & 10\\
					\hline
					4 & 0.75 & 30.75 & 17.77 & 31 & 10\\
					\hline
					5 & 0.75 & 30.75 & 18.5 & 31 & 10\\
					\hline
					6 & 0.75 & 30.75 & 20.03 & 31 & 10\\
					\hline
					7 & 0.75 & 30.75 & 18.43 & 31 & 10\\
					\hline
					8 & 0.75 & 31 & 17.68 & 31 & 10\\
					\hline
					9 & 0.75 & 30.75 & 17.21 & 31 & 10\\
					\hline
					10 & 0.75 & 30.75 & 16.94 & 31 & 10\\
					\hline
				\end{tabular} \\
				Rzeczywiste maksimum funkcji zostało znalezione już w 1 generacji.
			\subsection{Różne prawdopodobieństwa krzyżowania i mutacji}
				Eksperymenty przeprowadzone będą dla populacji 10 osobników.
				\subsubsection{Krzyżowanie: 1, Mutacja: 0.05}
				\begin{tabular}{|c|c|c|c|c|c|}
					\hline 
					iteracja &  min &  max & avg & global max &  X\\
					\hline
					1 & 0.75 & 30.75 & 15.97 & 30.75 & 9\\
					\hline
					2 & 0.75 & 30.75 & 15.15 & 30.75 & 11\\
					\hline
					3 & 0.75 & 30.75 & 8.75 & 30.75 & 9\\
					\hline
					4 & 0.75 & 30.75 & 15.95 & 30.75 & 9\\
					\hline
					5 & 0.75 & 30 & 16.1 & 30.75 & 9\\
					\hline
					6 & 0.75 & 30.75 & 15.7 & 30.75 & 9\\
					\hline
					7 & 0.75 & 30.75 & 17.43 & 30.75 & 9\\
					\hline
					8 & 0.75 & 30.75 & 17.12 & 30.75 & 9\\
					\hline
					9 & 0.75 & 30.75 & 18.73 & 30.75 & 9\\
					\hline
					10 & 10.75 & 31 & 19.0 & 31 & 10\\
					\hline
				\end{tabular} \\
				Rzeczywiste maksimum znaleziono w generacji 10.
				\subsubsection{Krzyżowanie: 0.8, Mutacja: 0.1}
				\begin{tabular}{|c|c|c|c|c|c|}
					\hline 
					iteracja &  min &  max & avg & global max &  X\\
					\hline
					1 & 0.75 & 30.75 & 13.95 & 30.75 & 11\\
					\hline
					2 & 0.75 & 30.75 & 14.38 & 30.75 & 11\\
					\hline
					3 & 0.75 & 30.75 & 17.98 & 30.75 & 11\\
					\hline
					4 & 6 & 30.75 & 20.4 & 30.75 & 11\\
					\hline
					5 & 15 & 30.75 & 22.43 & 30.75 & 11\\
					\hline
					6 & 0.75 & 30.75 & 18.15 & 30.75 & 11\\
					\hline
					7 & 0.75 & 30.75 & 20.02 & 30.75 & 11\\
					\hline
					8 & 0.75 & 30.75 & 15.64 & 30.75 & 11\\
					\hline
					9 & 0.75 & 30.75 & 22.02 & 30.75 & 11\\
					\hline
					10 & 0.75 & 30 & 15.22 & 30.75 & 11\\
					\hline
				\end{tabular}\\
				Rzeczywiste maksimum znaleziono w 20 generacji
				\subsubsection{Krzyżowanie: 0.5, Mutacja: 0.03}
				\begin{tabular}{|c|c|c|c|c|c|}
					\hline 
					iteracja &  min &  max & avg & global max &  X\\
					\hline
					1 & 6 & 30.75 & 22.95 & 30.75 & 9\\
					\hline
					2 & 6 & 30.75 & 22.48 & 30.75 & 9\\
					\hline
					3 & 6 & 30.75 & 20.05 & 30.75 & 9\\
					\hline
					4 & 6 & 30.75 & 20.55 & 30.75 & 9\\
					\hline
					5 & 0.75 & 31 & 17.7 & 31 & 10\\
					\hline
					6 & 0.75 & 30.75 & 21.1 & 31 & 10\\
					\hline
					7 & 0.75 & 30.75 & 18.73 & 31 & 10\\
					\hline
					8 & 0.75 & 30.75 & 19.52 & 31 & 10\\
					\hline
					9 & 0.75 & 30.75 & 18.73 & 31 & 10\\
					\hline
					10 & 10.75 & 30.75 & 19.73 & 31 & 10\\
					\hline
				\end{tabular} \\
				Rzeczywiste maksimum znaleziono w 5 generacji
	\section{Wnioski}
	Na zróżnicowanie wyników algorytmu genetycznego mają wpływ:
	\begin{itemize}
		\item Liczebność populacji -- ma ona bardzo duży wpływ na szybkość znalezienia wyniku. Im większa ona jest tym szybciej algorytm znajduje rozwiązanie.
		\item Prawdopodobieństwo mutacji -- wprowadza losowe zmiany między populacjami. Nie może być ona zbyt duża, gdyż wtedy algorytm działałby w sposób pseudolosowy
		\item Prawdopodobieństwo krzyżowania -- zwiększa zakres zmian w populacji.
	\end{itemize}
	
	
	
\end{document}